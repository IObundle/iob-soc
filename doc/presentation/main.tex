\documentclass [xcolor=svgnames, t] {beamer} 
\usepackage[utf8]{inputenc}
\usepackage{booktabs, comment} 
\usepackage[absolute, overlay]{textpos} 
\useoutertheme{infolines} 
\setbeamercolor{title in head/foot}{bg=internationalorange}
\setbeamercolor{author in head/foot}{bg=dodgerblue}
\usepackage{csquotes}
\usepackage[style=verbose-ibid,backend=bibtex]{biblatex}
\bibliography{bibfile}
\usepackage{amsmath}
\usepackage[makeroom]{cancel}
\usepackage{textpos}
\usepackage{tikz}
\usepackage{listings}
\graphicspath{ {./figures/} }


\usetheme{Madrid}
\definecolor{myuniversity}{RGB}{0, 60, 113}
\definecolor{internationalorange}{RGB}{231, 93,  42}
 	\definecolor{dodgerblue}{RGB}{0, 119,202}
\usecolortheme[named=myuniversity]{structure}

\graphicspath{ {./figures/} }

\title[IOB-SoC Presentation]{IOB-SoC}
\subtitle{A RISC-V based System on Chip}
\institute[IObundle Lda]{IObundle Lda.\\The Architecture for an Agile World}
\titlegraphic{\includegraphics[height=2.5cm]{Logo.png}}
\author[José T. de Sousa]{Jos\'e T. de Sousa}
\institute[IObundle Lda]{IObundle Lda}
\date{\today}


\addtobeamertemplate{navigation symbols}{}{%
    \usebeamerfont{footline}%
    \usebeamercolor[fg]{footline}%
    \hspace{1em}%
    \insertframenumber/\inserttotalframenumber
}

\begin{document}

\begin{frame}
 \titlepage   
\end{frame}

%%%%%%%%%%%%%%%%%%%%%%%%%%%%
\logo{\includegraphics[scale=0.2]{Logo.png}~%
}
%%%%%%%%%%%%%%%%%%%%%%%%%%

\begin{frame}{Outline}
\vspace{1cm}
\begin{center}
   \begin{itemize}
     \item Introduction
     \item Conclusion
 \end{itemize} 
\end{center}
\end{frame}


\begin{frame}{Introduction}
\begin{center}
    \begin{itemize}
      \item Building processor based systems from scratch is hard
      \item The IOB-SoC template eases this task
      \item Provides a base Verilog SoC equipped with
        \begin{itemize}
        \item a RISC-V CPU
        \item a memory system including a boot ROM and a RAM module
        \item a UART communication module
        \end{itemize}
      \item Users can add IP cores to build more complex SoCs
      \item Here, the addition of a timer IP is exemplified
    \end{itemize}
\end{center}
\end{frame}

\begin{frame}{Project setup}
\begin{center}
  \begin{itemize}
    \item Use a Linux machine virtual or not
    \item Make sure you can access {\it github.io} and {\it bitbucket.org} using ssh keys
    \item Go to {\it bitbucket.org/jjts/iob-soc}
    \item Fork the repository to create your own remote repository
    \item Create a directory for your project\\
      {\tt >mkdir mysoc \&\& cd mysoc}
    \item Follow the instructions in the README to clone the repository into your project directory
  \end{itemize}
\end{center}
\end{frame}


\begin{frame}{Project editing}
\begin{itemize}
\item Your SoC will be inspired in IOB-SoC, not superimposed on IOB-SoC
\item Do not edit any files in the iob-soc directory (clone) unless you want to contribute to it
\item If you want to submit fixes or improvements to the origina iob-soc repository:
  \begin{itemize}
  \item Commit and push your changes to your fork
  \item Submbit a Pull Request to the original iob-soc repository
  \end{itemize}
\item If you have benefited from IOB-SoC let others benefit from your improvements
\end{itemize}
\end{frame}

\begin{frame}{Add the timer hardware}
\begin{itemize}
\item Create the timer hardware with a CPU interface
\item Alternatively download an exsting one from
  \\{\tt git clone git@bitbucket.org:jjts/iob-timer.git}
\item Copy the RTL directory from the iob-soc clone:
  \\{\tt cp -r iob-soc/rtl .}
\item Edit the system header file {\tt ./rtl/include/system.vh} as in next slide\\
\end{itemize}
\end{frame}

\begin{frame}[fragile]{The edited file {\tt system.vh}}
\begin{tiny}
\begin{lstlisting}
//
// HARDWARE DEFINITIONS
//

//Optional memories (passed as command line macro)
`define USE_BOOT
`define USE_DDR

// slaves
// minimum 3 slaves: boot, uart and reset
// 1 slave for bootram and/or main RAM memory
// DDR needs 2 slaves: cache and cache controller
// 1 ***new*** slave for timer
`define N_SLAVES 7

//bits reserved to identify slave (2**N_SLAVES-1 is reserved)
`define N_SLAVES_W 3

//peripheral address prefixes
`define BOOT_BASE 0
`define UART_BASE 1
`define SOFT_RESET_BASE 2
`define MAINRAM_BASE 3
`define CACHE_BASE 4
`define CACHE_CTRL_BASE 5
//***new*** base for timer
`define TIMER_BASE 6
...
\end{lstlisting}
\end{tiny}
\end{frame}

\begin{frame}[fragile]{Instantiate the timer in file {\tt rtl/src/system.v}}
\begin{tiny}
\begin{lstlisting}
`timescale 1 ns / 1 ps
`include "system.vh"

module system (
               input                clk,
               input                reset,
               ...
               ...
               ...

   //***new*** timer instance
   time_counter #(.COUNTER_WIDTH(32))
   timer (
          rst(rst),
          clk(clk),
          .addr(addr),
          .data_in(m_wdata),
          .data_out(s_rdata[`TIMER_BASE]),
          .valid(s_valid[`TIMER_BASE]),
          .ready(s_ready[`TIMER_BASE])
    );

endmodule

\end{lstlisting}
\end{tiny}
\end{frame}

\begin{frame}{Add the timer software}
\begin{itemize}
\item Copy the software directory from the iob-soc clone:
  \\{\tt cp -r iob-soc/software .}
\item Edit the {\tt ./software/firmware/firmware.c} file as in next slide\\
\end{itemize}
\end{frame}

\begin{frame}[fragile]{Edit file {\tt firmware.c} to use the timer}
\begin{tiny}
\begin{lstlisting}
#include "system.h"
#include "iob-uart.h"

#define UART (UART_BASE<<(DATA_W-N_SLAVES_W))
#define SOFT_RESET (SOFT_RESET_BASE<<(ADDR_W-N_SLAVES_W))

int main()
{ 

  //***new*** variable to read timer cycle count
  unsigned long time = timer_get_count(TIMER_BASE);
  
  uart_init(UART,UART_CLK_FREQ/UART_BAUD_RATE);   

  uart_printf("Hello world!\n");

  uart_txwait();
  
  //***new*** code section to read current timer count and compute 
  //and print elapsed clock cycles
  time =  timer_get_count(TIMER_BASE) - time;

  uart_printf("Total execution time: %d\n", (time*10000000)/UART_CLK_FREQ);
                          
  //uncomment the below to reset and start from boot ram
  //MEMSET(SOFT_RESET, 0, 3);

  return 0;
}
\end{lstlisting}
\end{tiny}
\end{frame}


\begin{frame}[fragile]{Edit Makefile to compile timer}
\begin{tiny}
\begin{lstlisting}
TOOLCHAIN_PREFIX = riscv32-unknown-elf-
PYTHON_DIR := ../python/
UART_DIR := ../../submodules/iob-uart/c-driver

#***new*** specify timer directory and addi it to include path 
TIMER_DIR := ../../iob-timer
INCLUDE = -I. -I$(UART_DIR) -I$(TIMER_DIR)
DEFINE = -DUART_BAUD_RATE=$(BAUD) -DUART_CLK_FREQ=$(FREQ)

#***new*** add timer.c to the source list
SRC = firmware.c $(UART_DIR)/iob-uart.c firmware.S

all: firmware.hex

firmware.hex: firmware.lds $(SRC) system.h $(UART_DIR)/iob-uart.h
	$(TOOLCHAIN_PREFIX)gcc -Os -ffreestanding  -nostdlib -o firmware.elf $(DEFINE) $(INCLUDE) $(SRC) --std=gnu99 -Wl,-Bstatic,-T,firmware.lds,-Map,firmware.map,--strip-debug -lgcc -lc
	$(TOOLCHAIN_PREFIX)objcopy -O binary firmware.elf firmware.bin
	$(eval MEM_SIZE=`./get_firmsize.sh`)
	$(PYTHON_DIR)/makehex.py firmware.bin $(MEM_SIZE) > progmem.hex
	$(eval MEM_SIZE=`$(PYTHON_DIR)/get_memsize.py MAINRAM_ADDR_W`)
	$(PYTHON_DIR)/makehex.py firmware.bin $(MEM_SIZE) > firmware.hex

system.h: ../../rtl/include/system.vh
	sed s/\`/\#/g ../../rtl/include/system.vh > system.h

clean:
	@rm -rf firmware.bin firmware.elf firmware.map *.hex *.dat
	@rm -rf uart_loader system.h
	@rm -rf ../uart_loader

.PHONY: all clean
\end{lstlisting}
\end{tiny}
\end{frame}



\begin{frame}[fragile]{Edit simulation Makefile}

\begin{itemize}
\item Copy the simulation folder from the iob-soc repository\\
  {\tt cp -r iob-soc/simulation} 
\item Edit the Makefile to compile timer hardware as below
\end{itemize}

\begin{tiny}
\begin{lstlisting}
#simulation baud rate
BAUD := 1000000
FREQ := 100000000

#file paths
#***new*** add iob-soc directory to relevant paths
IOB_SOC_DIR := ../../iob-soc
FIRM_DIR := ../../$(IOB_SOC_DIR)/software/firmware
BOOT_DIR := ../../software/bootloader
RTL_DIR := ../../rtl
RISCV_DIR := ../../$(IOB_SOC_DIR)/submodules/iob-rv32
UART_DIR := ../../$(IOB_SOC_DIR)/submodules/iob-uart
FIFO_DIR := ../../$(IOB_SOC_DIR)/submodules/fifo
PYTHON_DIR := ../../$(IOB_SOC_DIR)/software/python
CACHE_DIR := ../../$(IOB_SOC_DIR)/submodules/iob-cache
AXI_RAM_DIR := ../../$(IOB_SOC_DIR)/submodules/axi-mem
#***new*** add timer directory  and update hw include paths
TIMER_DIR := ../../iob-timer
#hw includes
HW_INCLUDE := -I. -I$(RTL_DIR)/include -I$(UART_DIR)/rtl/include -I$(CACHE_DIR)/rtl/header -I$(TIMER_DIR)
(...)
#hardware sources
#***new*** add timer verilog source 
VSRC = $(RTL_DIR)/testbench/system_tb.v  $(RTL_DIR)/src/*.v $(RTL_DIR)/src/memory/behav/*.v $(RISCV_DIR)/picorv32.v $(UART_DIR)/rtl/src/iob-uart.v $(FIFO_DIR)/afifo.v $(CACHE_DIR)/rtl/src/gen_mem_reg.v $(CACHE_DIR)/rtl/src/iob-cache.v $(AXI_RAM_DIR)/rtl/axi_ram.v $(TIMER_DIR)/rtl/timer.v
\end{lstlisting}
\end{tiny}
\end{frame}


\begin{frame}{Conclusion}
\begin{itemize}
\item 
\item 
\item 
\end{itemize}
\end{frame}




%\begin{frame}{Introduction}
%\begin{center}
%  \begin{columns}[onlytextwidth]
%    \column{0.5\textwidth}
%    bla
%    \column{0.5\textwidth}
%    \begin{figure}
%      \centering
%       \includegraphics[width=0.9\textwidth]{turb.jpg}
%      \caption{1. Flow visualisation (source: www.bronkhorst.com).}
%      \label{fig:my_label}
%    \end{figure}
%  \end{columns}
%\end{center}
%\end{frame}

\end{document}
