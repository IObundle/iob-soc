Table~\ref{tab:confs} describes the IP core configuration. The core may be configured using macros or parameters:

\begin{description}
    \item \textbf{'M'} Macro: a Verilog macro or \texttt{define} directive is used to include or exclude code segments, to create core configurations that are valid for all instances of the core.
\item \textbf{'P'} Parameter: a Verilog parameter is passed to each instance of the core and defines the configuration of that particular instance.
\end{description}

\begin{xltabular}{\textwidth}{|l|c|c|c|c|X|} \hline
    \rowcolor{iob-green}
    {\bf Configuration} & {\bf Type} & {\bf Min} & {\bf Typical} & {\bf Max} & {\bf Description}
    \\ \hline \hline
    \input confs_tab
    \caption{Core Configuration.}\label{tab:confs}
\end{xltabular}

The macros not listed above are constants. They improve the code readability and
should not be changed by the user. These constants are listed below:
\input constants

The top-level parameters not listed above are constant or derived from the primary parameters. They improve the code readability and should not be changed by the user. These parameters are listed below:
\input derived_params

