\documentclass{ug}
\usepackage{xltabular}

%replace ipcore-name by the name of your ip core (e.g. IOb-Cache) and description by a brief description (e.g. a Configurable Cache)

\title{%
\Huge IOB-UART \\
 \vspace*{3cm}
\Large A RISC-V UART
}

\header{IOB-UART, a RISC-V UART}

\date{\today}
\category{User Guide, \input{\NAME_version.tex}, Build \input{shortHash.tex}}

\input{color}

\begin{document}

\maketitle
\pagenumbering{gobble}

\vspace*{\fill}
User Guide, \input{\NAME_version.tex}, Build \input{shortHash.tex}
\hspace*{\fill} \includegraphics[keepaspectratio,scale=.7]{Logo.png}

\cleardoublepage
\pagenumbering{roman}
\setcounter{page}{1}
\today & Initial document version \input{version}. \\ \hline

\cleardoublepage
\tableofcontents
\clearpage
\listoftables
\clearpage
\listoffigures
\cleardoublepage
\pagenumbering{arabic}
\setcounter{page}{1}
\section{Introduction}
\label{sec:intro}
IOb-SoC is a RISC-V-based System-on-Chip Platform written in Verilog, which
users can download for free, modify, simulate and implement in FPGA or ASIC. It
supports stand-alone and boot loading modes, and can use an internal RAM or an
external DDR controller via an L1/L2 cache system. The IP is currently supported
in ASICs and FPGAs. Licensable commercial versions are available.

\input{symb}

\subsection{Features}
\label{sec:feat}
\begin{itemize}
\item Supported in IObundle's RISC-V IOb-SoC open-source and free of charge template.
\item IObundle's IOb-SoC native CPU interface.
\item Verilog basic UART implementation.
\item Soft reset and enable functions.
\item Runtime configurable baud rate
\item C software driver at the bare-metal level.
\item Simple Verilog testbench for the IP's {\em nucleus}.
\item System-level Verilog testbench available when simulating the IP embedded in IOb-SoC.
\item Simulation Makefile for the open-source and free of charge Icarus Verilog simulator.
\item FPGA synthesis and implementation scripts for two FPGA families from two FPGA vendors.
\item Automated creation of FPGA netlists
\item Automated production of documentation using the open-source and free Latex framework.
\item IP data automatically extracted from FPGA tool logs to include in documents.
\item Makefile tree for full automation of simulation, FPGA implementation and document production.
\item AXI4 Lite CPU interface (premium option).
\item Parity bits  (premium option).
\end{itemize}


\subsection{Benefits}
\label{sec:benef}
\begin{itemize}
  \itemsep-0.5em
\item Compact and easy to integrate hardware and software implementation
\item Can fit many instances in low cost FPGAs and ASICs
\item Low power consumption
\end{itemize}


\subsection{Deliverables}
\label{sec:deliv}
\begin{itemize}
  \itemsep-0.5em
\item ASIC or FPGA synthesized netlist or Verilog source code, and respective
  synthesis and implementation scripts
\item ASIC or FPGA verification environment by simulation and emulation
\item Bare-metal software driver and example user software
\item User documentation for easy system integration
\item Example integration in IOb-SoC (optional)
\end{itemize}


\section{Description}

This section gives a detailed description of the IP core. The high-level block
diagram is presented, along with a description of its blocks. The parameters and
macros that define the core configuration are listed and explained. The
interface signals are enumerated and described; if timing diagrams are needed,
they are shown after the interface signals. Finally the Control and Status
Registers (CSR) are outlined and explained.

\subsection{Block Diagram}
\label{sec:bdd}
\input{bdd}

\ifdefined\CONFS
\subsection{Configuration}
\label{sec:ipconfig}
Table~\ref{tab:confs} describes the IP core configuration. The core may be configured using macros or parameters:

\begin{description}
\item \textbf{'M'} Macro: a Verilog macro or ``\`define'' directive is used to include or exclude or code segments, to create core configurations that are valid for all instances of the core.
\item \textbf{'P'} Parameter: a Verilog parameter is passed to each instance of the core and defines the configuration of that particular instance.
\end{description}

\begin{xltabular}{\textwidth}{|l|c|c|c|c|X|} \hline
    \rowcolor{iob-green}
    {\bf Configuration} & {\bf Type} & {\bf Min} & {\bf Typical} & {\bf Max} & {\bf Description}
    \\ \hline \hline
    \input confs_tab
    \caption{Core Configuration.}\label{tab:confs}
\end{xltabular}

The parameters in the top-level Verilog module that are not listed above are
called Derived Parameters. They are given as function of the primary parameters
and should never be changed. They are used to simplify the definition of the
interface and internal signals. The list of derived parameters is given below:
\input derived_params

\fi

\subsection{Interface Signals}
\label{sec:ifsig}
\begin{table}[H]
  \centering
  \begin{tabular}{|l|l|r|p{10.5cm}|}

    \hline
    \rowcolor{iob-green}
    {\bf Name} & {\bf Direction} & {\bf Width} & {\bf Description}  \\ \hline \hline

    \input{gen_if_tab}

  \end{tabular}
  \caption{General Interface Signals.}
  \label{gen_if_tab:is}
\end{table}

\begin{table}[H]
  \centering
  \begin{tabular}{|l|l|r|p{8.5cm}|}

    \hline
    \rowcolor{iob-green}
    {\bf Name} & {\bf Direction} & {\bf Width} & {\bf Description}  \\ \hline \hline

    \input{iob_s_if_tab}

  \end{tabular}
  \caption{IObundle Interface Signals}
  \label{tab:if_iob_s}
\end{table}

\begin{table}[H]
  \centering
  \begin{tabular}{|l|l|r|p{9.5cm}|}
    
    \hline
    \rowcolor{iob-green}
    {\bf Name} & {\bf Direction} & {\bf Width} & {\bf Description}  \\ \hline \hline

    \input{rs232_if_tab}
 
  \end{tabular}
  \caption{RS232 Interface Signals}
  \label{tab:if_rs232}
\end{table}

%TODO
%\input{\TEX/ug/axil_s_if}


%timing diagrams
\ifdefined\TD
\subsection{Timing Diagrams}
\label{sec:td}
\input{td}
\fi

%software components
\ifdefined\SWREG
\subsection{Control and Status Registers}
\label{sec:swreg}
\input{swreg}
\fi

\section{Usage}

\subsection{Instantiation}
\label{sec:inst}
The IOb-UART is a fully synchronous, single clock ({\tt clk}) domain
design with an asynchronous active-high reset signal ({\tt arst}) that drives
all the flip-flops in the design. The reset signal should be de-asserted
synchronously with the clock signal's rising edge.

The IOb-UART works attached to a CPU core, using the IOb Native Bus
interface. It outputs (pin {\tt txd}) and receives (pin {\tt rxd}) an
RS232-encoded serial data stream. The RS232 protocol also specifies and
handshking signal pair consisting of the signals {\tt cts} and {\tt rts}.



\subsection{Simulation}
\label{sec:tbbd}
The provided testbench implements a self-loop, where the IOb-UART handshakes
({\tt cts} to {\tt rts}, and sends data to itself ({\tt txd} to {\tt rxd}). The
testbench drives the clock and reset signals, and emulates the CPU actions with
a simple control block.


\ifdefined\ASICSYNTH
\subsection{ASIC Synthesis}
\label{sec:synth}
This IP core synthesizes for FPGA and ASIC with very low logic resources
consumption. The implementation does not require memory or arithmetic resources.

\fi

\ifdefined\FPGACOMP
\subsection{FPGA Compilation}
\label{sec:fpga}
\input{fpga}
\fi

\ifdefined\RESULTS
\section{Implementation Results}
\label{sec:results}
The results obtained use all the default user configurable parameters, which affect the RAM usage results.

\fi

\ifdefined\CUSTOM
\input{custom}
\fi

\end{document}
