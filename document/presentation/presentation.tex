\documentclass [xcolor=svgnames, t] {beamer} 
\usepackage[utf8]{inputenc}
\usepackage[T1]{fontenc}
\usepackage{booktabs, comment} 
\usepackage[absolute, overlay]{textpos} 
\useoutertheme{infolines} 
\setbeamercolor{title in head/foot}{bg=internationalorange}
\setbeamercolor{author in head/foot}{bg=dodgerblue}
\usepackage{csquotes}
\usepackage[style=verbose-ibid,backend=bibtex]{biblatex}
\bibliography{bibfile}
\usepackage{amsmath}
\usepackage[makeroom]{cancel}
\usepackage{textpos}
\usepackage{tikz}
\usepackage{listings}
\graphicspath{ {./figures/} }
\usepackage{hyperref}
\hypersetup{
    colorlinks=true,
    linkcolor=blue,
    filecolor=magenta,      
    urlcolor=cyan,
}

\usetheme{Madrid}
\definecolor{myuniversity}{RGB}{0, 60, 113}
\definecolor{internationalorange}{RGB}{231, 93,  42}
 	\definecolor{dodgerblue}{RGB}{0, 119,202}
\usecolortheme[named=myuniversity]{structure}

\graphicspath{ {./figures/} }

\title[IOB-SoC Presentation]{IOB-SoC}
\subtitle{Tutorial: Create a RISC-V-based System on Chip}
\institute[IObundle Lda]{IObundle Lda.}
\titlegraphic{\includegraphics[height=2.5cm]{Logo.png}}
%\author[José T. de Sousa]{Jos\'e T. de Sousa}
%\institute[IObundle Lda]{IObundle Lda}
\date{\today}


\addtobeamertemplate{navigation symbols}{}{%
    \usebeamerfont{footline}%
    \usebeamercolor[fg]{footline}%
    \hspace{1em}%
    \insertframenumber/\inserttotalframenumber
}

\begin{document}

\begin{frame}
 \titlepage   
\end{frame}

%%%%%%%%%%%%%%%%%%%%%%%%%%%%
\logo{\includegraphics[scale=0.2]{Logo.png}~%
}
%%%%%%%%%%%%%%%%%%%%%%%%%%

\begin{frame}{Outline}
\begin{center}
   \begin{itemize}
     \item Introduction
     \item Project setup
     \item Instantiate an IP core in your SoC
     \item Edit file {\tt firmware.c} to drive the new peripheral
     \item Simulate IOb-SoC
     \item Run IOb-SoC in FPGA
     \item Conclusions and future work
 \end{itemize}
\end{center}
\end{frame}


\begin{frame}{Introduction}
\begin{center}
    \begin{itemize}
      \item Building processor-based systems from scratch is challenging
      \item The IOB-SoC template eases this task
      \item Provides a base Verilog SoC equipped with
        \begin{itemize}
        \item a RISC-V CPU
        \item a memory system including boot ROM, RAM, cache system and AXI4 interface to DDR
        \item a UART communications module
        \end{itemize}
      \item Users can add IP cores and software to build more complex SoCs
      \item Here, the addition of a timer IP core and its software driver is exemplified
    \end{itemize}
\end{center}
\end{frame}

\begin{frame}{Project setup}
\begin{center}
  \begin{itemize}
    \item Use a Linux machine or VM
    \item Install the latest {\bf stable} version of the open source Icarus Verilog simulator (\url{iverilog.icarus.com})
    \item Make sure you have {\bf ssh} push/pull access to \url{github.com} (htps access is not enough)
      key
    \item Visit the repository at \url{github.com/IObundle/iob-soc}
    \item Follow the instructions in its README file to clone the repository ans install the tools
  \end{itemize}
\end{center}
\end{frame}


\begin{frame}{Instantiate an IP core in your SoC}
  \begin{itemize}
  \item The Timer IP core at \url{github.com/IObundle/iob-timer.git} will be used as an example
  \item Add the Timer IP core repository as a git submodule of your IOb-SoC repository:\\
    {\tt \tiny git submodule add git@github.com:IObundle/iob-timer.git submodules/TIMER}
  \item Add the timer IP core to the list of peripherals in the {\tt ./system.mk} file:\\
    {\tt PERIPHERALS:=UART {\em TIMER}}
  \item An IP core can be integrated into IOb-SoC if it provides the following files: 
    \begin{itemize}
    \item hardware/hardware.mk
    \item software/embedded.mk
    \end{itemize}
  \item Study these files and its references in the Timer IP core repository.
  \end{itemize}
\end{frame}

\begin{frame}[fragile]{Edit the {\tt firmware.c} file to look as follows and drive the new peripheral}
  {\tt ./software/firmware/firmware.c}
  \begin{tiny}
    \begin{lstlisting}
#include "system.h"
#include "periphs.h"
#include "iob-uart.h"
#include "iob_timer.h"

int main()
{
  unsigned long long elapsed;
  unsigned int elapsedu;

  //init timer and uart
  timer_init(TIMER_BASE);
  uart_init(UART_BASE, FREQ/BAUD);

  printf("\nHello world!\n");
  
  uart_txwait();

  //read current timer count, compute elapsed time
  elapsed  = timer_get_count(TIMER_BASE);
  elapsedu = timer_time_us(TIMER_BASE);

  printf("\nExecution time: %d clock cycles\n", (unsigned int) elapsed);
  printf("\nExecution time: %dus @%dMHz\n\n", elapsedu, FREQ/1000000);

  uart_txwait();
}
    \end{lstlisting}
  \end{tiny}
\end{frame}


\begin{frame}[fragile]{Simulate IOb-SoC}
\begin{itemize}
\item The following assumes the Icarus Verilog simulator is installed locally and that the SIMULATOR variable is set as SIMULATOR=icarus (see the README.md file)
\item To run the simulation with the firmware pre-initialised in the memory:\\
  {\tt make sim}
\item The firmware and bootloader C files, and the system's Verilog files are compiled as you can see from the printed messages
\item During the simulation itself, the following is printed:
\end{itemize}

\begin{tiny}
  \begin{lstlisting}

IOb-Bootloader: USE_DDR=0 RUN_DDR=0
IOb-Bootloader: Restart CPU to run user program...

Hello world!

Execution time: 2190 clock cycles

Execution time: 23us @100MHz

  \end{lstlisting}
\end{tiny}
\end{frame}


\begin{frame}{Run IOb-SoC in FPGA}
\begin{itemize}
\item To compile and run your SoC in one of our FPGA boards, contacts us at info
  @ iobundle . com.
\item To add your own FPGA board, add a directory into {\tt ./hardware/fpga},
  using the existing board directories as examples
\item To compile and run the FPGA design for a set of parameters:\\
  {\tt make test-board-config INIT\_MEM=0 RUN\_DDR=1}\\
  This will compile the software and the hardware, produce an FPGA image, load the hardware and firmware to the board, and direct the standard output to your PC.
\item To recompile and sent only the firmware to the board via UART:\\
  {\tt make board-run}
\end{itemize}
\end{frame}

\begin{frame}{Conclusions}  
  \begin{itemize}
  \item A tutorial on creating a simple SoC using IOb-SoC has been presented
  \item The addition of an example peripheral IP core has been illustrated
  \item A simple software driver for the IP core has been described
  \item Simulation of IOb-SoC hs been explained
  \item Compiling and running IOb-SoC on FPGA has been explained
  \end{itemize}
\end{frame}

% for including figures in slides:
%\begin{frame}{Introduction}
%\begin{center}
%  \begin{columns}[onlytextwidth]
%    \column{0.5\textwidth}
%    bla
%    \column{0.5\textwidth}
%    \begin{figure}
%      \centering
%       \includegraphics[width=0.9\textwidth]{turb.jpg}
%      \caption{1. Flow visualisation (source: www.bronkhorst.com).}
%      \label{fig:my_label}
%    \end{figure}
%  \end{columns}
%\end{center}
%\end{frame}

\end{document}
