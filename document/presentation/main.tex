\documentclass [xcolor=svgnames, t] {beamer} 
\usepackage[utf8]{inputenc}
\usepackage[T1]{fontenc}
\usepackage{booktabs, comment} 
\usepackage[absolute, overlay]{textpos} 
\useoutertheme{infolines} 
\setbeamercolor{title in head/foot}{bg=internationalorange}
\setbeamercolor{author in head/foot}{bg=dodgerblue}
\usepackage{csquotes}
\usepackage[style=verbose-ibid,backend=bibtex]{biblatex}
\bibliography{bibfile}
\usepackage{amsmath}
\usepackage[makeroom]{cancel}
\usepackage{textpos}
\usepackage{tikz}
\usepackage{listings}
\graphicspath{ {./figures/} }
\usepackage{hyperref}
\hypersetup{
    colorlinks=true,
    linkcolor=blue,
    filecolor=magenta,      
    urlcolor=cyan,
}

\usetheme{Madrid}
\definecolor{myuniversity}{RGB}{0, 60, 113}
\definecolor{internationalorange}{RGB}{231, 93,  42}
 	\definecolor{dodgerblue}{RGB}{0, 119,202}
\usecolortheme[named=myuniversity]{structure}

\graphicspath{ {./figures/} }

\title[IOB-SoC Presentation]{IOB-SoC}
\subtitle{A RISC-V-based System on Chip}
\institute[IObundle Lda]{IObundle Lda.\\The Architecture for an Agile World}
\titlegraphic{\includegraphics[height=2.5cm]{Logo.png}}
%\author[José T. de Sousa]{Jos\'e T. de Sousa}
\institute[IObundle Lda]{IObundle Lda}
\date{\today}


\addtobeamertemplate{navigation symbols}{}{%
    \usebeamerfont{footline}%
    \usebeamercolor[fg]{footline}%
    \hspace{1em}%
    \insertframenumber/\inserttotalframenumber
}

\begin{document}

\begin{frame}
 \titlepage   
\end{frame}

%%%%%%%%%%%%%%%%%%%%%%%%%%%%
\logo{\includegraphics[scale=0.2]{Logo.png}~%
}
%%%%%%%%%%%%%%%%%%%%%%%%%%

\begin{frame}{Outline}
\begin{center}
   \begin{itemize}
     \item Introduction
     \item Project setup
     \item Create an IP core to instantiate in your SoC
     \item Edit the ./system.mk configuration file to declare a new peripheral
     \item Edit file {\tt firmware.c} to drive the new peripheral
     \item Run the firmware in internal SRAM
     \item Run the firmware in external DDR
     \item Simulate and implement the system
     \item Implement in FPGA
     \item Implement in ASIC (WIP)
     \item Conclusions and future work
 \end{itemize}
\end{center}
\end{frame}


\begin{frame}{Introduction}
\begin{center}
    \begin{itemize}
      \item Building processor-based systems from scratch is challenging
      \item The IOB-SoC template eases this task
      \item Provides a base Verilog SoC equipped with
        \begin{itemize}
        \item a RISC-V CPU
        \item a memory system including boot ROM, RAM and AXI4 interface to DDR
        \item a UART communications module
        \end{itemize}
      \item Users can add IP cores and software to build more complex SoCs
      \item Here, the addition of a timer IP and its software driver is exemplified
    \end{itemize}
\end{center}
\end{frame}

\begin{frame}{Project setup}
\begin{center}
  \begin{itemize}
    \item Use a Linux machine or VM
    \item Install the latest stable version of the open source Icarus Verilog simulator (\url{iverilog.icarus.com})
    \item Make sure you can access \url{github.com} using an ssh key
    \item Clone the repository \url{github.com/IObundle/iob-soc}
    \item Follow the instructions in its README file
  \end{itemize}
\end{center}
\end{frame}


\begin{frame}{Create an IP core to instantiate in your SoC}
  \begin{itemize}
  \item Create a timer IP core repository or, alternatively, use the one at \url{www.github.com/IObundle/iob-timer.git}
  \item An IP core can be integrated into IOb-SoC if it provides the following files: 
    \begin{enumerate}
    \item hardware/hardware.mk
    \item software/software.mk
    \end{enumerate}
    \begin{itemize}
      \item[--] Please refer to the files {\tt hardware/hardware.mk} and {\tt software/software.mk} in the iob-timer submodule to learn how to organize a peripheral core.
    \end{itemize}
  \item Add the IP core repository as a git submodule of your IOb-SoC repository:\\
    {\tiny \tt git submodule add https://github.com/IObundle/iob-timer.git submodules/TIMER}
  \item To configure the system to host the IP core, edit the {\tt ./system.mk} file as in the next slide
  \end{itemize}
\end{frame}

% Following line allows for a more copy-paste-able code
\lstset{basicstyle=\ttfamily,columns=fullflexible}

\begin{frame}[fragile]{Edit the {\tt ./system.mk} configuration file to configure the system with a new peripheral}
\begin{tiny}
\begin{lstlisting}
#FIRMWARE
FIRM_ADDR_W:=13

#SRAM
SRAM_ADDR_W=13

#DDR
ifeq ($(USE_DDR),)
	USE_DDR:=0
endif
ifeq ($(RUN_DDR),)
	RUN_DDR:=0
endif

DDR_ADDR_W:=30
CACHE_ADDR_W:=24

#ROM
BOOTROM_ADDR_W:=12

#Init memory (only works in simulation or in FPGA)
ifeq ($(INIT_MEM),)
	INIT_MEM:=0
endif

#Peripheral list (must match respective submodule or folder name in the submodules directory)
PERIPHERALS:=UART


\end{lstlisting}
\end{tiny}
\end{frame}


\begin{frame}[fragile]{Edit the {\tt ./system.mk} configuration file to configure the system with a new peripheral (continued: 2)}
\begin{tiny}
\begin{lstlisting}

#SIMULATION TEST
SIM_LIST="SIMULATOR=icarus" "SIMULATOR=ncsim"
#SIM_LIST="SIMULATOR=icarus"
LOCAL_SIM_LIST=icarus #leave space in the end

ifeq ($(SIMULATOR),ncsim)
	SIM_SERVER=$(SIM_USER)@micro7.lx.it.pt
	SIM_USER=user19
else
#default
	SIMULATOR:=icarus
endif

\end{lstlisting}
\end{tiny}
\end{frame}

\begin{frame}[fragile]{Edit the {\tt ./system.mk} configuration file to configure the system with a new peripheral (continued: 3)}
\begin{tiny}
\begin{lstlisting}
#BOARD TEST
BOARD_LIST="BOARD=CYCLONEV-GT-DK" "BOARD=AES-KU040-DB-G"
#BOARD_LIST="BOARD=AES-KU040-DB-G"
#BOARD_LIST="BOARD=CYCLONEV-GT-DK"
#LOCAL_BOARD_LIST=CYCLONEV-GT-DK #leave space in the end
#LOCAL_COMPILER_LIST=CYCLONEV-GT-DK AES-KU040-DB-G

ifeq ($(BOARD),AES-KU040-DB-G)
	COMPILE_USER=$(USER)
	COMPILE_SERVER=$(COMPILE_USER)@pudim-flan.iobundle.com
	COMPILE_OBJ=synth_system.bit
	BOARD_USER=$(USER)
	BOARD_SERVER=$(BOARD_USER)@baba-de-camelo.iobundle.com
else ifeq ($(BOARD),CYCLONEV-GT-DK)
	COMPILE_SERVER=$(COMPILE_USER)@pudim-flan.iobundle.com
	COMPILE_USER=$(USER)
	COMPILE_OBJ=output_files/top_system.sof
	BOARD_SERVER=$(BOARD_USER)@pudim-flan.iobundle.com
	BOARD_USER=$(USER)
else
#default
	BOARD=CYCLONEV-GT-DK
	COMPILE_OBJ=output_files/top_system.sof
endif

\end{lstlisting}
\end{tiny}
\end{frame}

\begin{frame}[fragile]{Edit the {\tt ./system.mk} configuration file to configure the system with a new peripheral (continued: 4)}
\begin{tiny}
\begin{lstlisting}

#ROOT DIR ON REMOTE MACHINES
REMOTE_ROOT_DIR=./sandbox/iob-soc

#ASIC
ASIC_NODE:=umc130

#DOC TYPE
DOC_TYPE:=presentation
\end{lstlisting}
\end{tiny}
\end{frame}


\begin{frame}[fragile]{Edit the {\tt firmware.c} file to drive the new peripheral}
  {\tt ./software/firmware/firmware.c}
  \begin{tiny}
    \begin{lstlisting}
#include "system.h"
#include "periphs.h"
#include "iob-uart.h"
#include "iob_timer.h"

int main()
{
  unsigned long long elapsed;
  unsigned int elapsedu;

  //read current timer count, compute elapsed time
  elapsed  = timer_get_count(TIMER_BASE);
  elapsedu = timer_time_us(TIMER_BASE);

  //init uart 
  uart_init(UART_BASE, FREQ/BAUD);

  uart_printf("\nHello world!\n");
  
  uart_txwait();

  uart_printf("\nExecution time: %d clocks in %dus @%dMHz\n\n", 
              (unsigned int)elapsed, elapsedu, FREQ/1000000);

  uart_txwait();
  return 0;
}
\end{lstlisting}
\end{tiny}
\end{frame}


\begin{frame}{Run the firmware in internal SRAM}
\begin{enumerate}
\item Initialize the SRAM with the firmware
  \begin{itemize}
  \item Define {\tt INIT\_MEM=1}
  \item Works in simulation and FPGA
  \item Firmware may be recompiled and reloaded via UART afterwords
  \end{itemize}
\item Do not initialize the SRAM with the firmware (default)
  \begin{itemize}
  \item Works in simulation, FPGA and ASIC
  \item The firmware is (re)compiled and (re)loaded via UART
  \end{itemize}
\end{enumerate}
\end{frame}

\begin{frame}{Run the firmware in external DDR}
\begin{enumerate}
\item Initialize the DDR with the firmware
  \begin{itemize}
  \item Define {\tt USE\_DDR=1} and {\tt INIT\_MEM=1}
  \item Works in simulation only
  \item In FPGA or ASIC the external DDR cannot be initialized
  \end{itemize}
\item Do not initialize the DDR with the firmware (default)
  \begin{itemize}
  \item Define {\tt USE\_DDR=1}
  \item This option is valid for simulation, FPGA and ASIC
  \item The firmware is (re)compiled and (re)loaded via UART
  \item In FPGA or ASIC a third party DDR controller IP core is required
  \end{itemize}
\end{enumerate}
\end{frame}


\begin{frame}[fragile]{Simulate IOb-SoC}
  \begin{itemize}
  \item Simulation runs in directory {\tt ./hardware/simulation} pointed by the
    SIMULATOR variable in system.mk
  \item To add your simulation directory in {\tt ./hardware/simulation} for your
    simulator use the files in the existing simulation directories as examples.
  \item In file {\tt ./system.mk} find and insert in the following section as given:
%   \begin{tiny}
    \begin{lstlisting}
ifeq ($(SIMULATOR),ncsim)
	SIM_SERVER=$(SIM_USER)@micro7.lx.it.pt
	SIM_USER=user19
else ifeq ($(SIMULATOR),your_simulator)
	SIM_SERVER=$(SIM_USER)@your.simulation.server
	SIM_USER=your_username
else
#default
	SIMULATOR:=icarus
endif
    \end{lstlisting}
%    \end{tiny}
  \end{itemize}
\end{frame}


\begin{frame}[fragile]{Simulate IOb-SoC (continued)}
\begin{itemize}
  \item To run the simulation type {\tt make sim INIT\_MEM=1}
  \item The firmware, bootloader and system verilog description are compiled as you can see from the printed messages
  \item During simulation the following is printed:
\end{itemize}
\begin{tiny}
\begin{lstlisting}

IOb-SoC Bootloader:

Reboot CPU and run program...

Hello world!

Execution time: 6583 clocks in 66us @100MHz

\end{lstlisting}
\end{tiny}
\end{frame}


\begin{frame}{Implement in FPGA}
\begin{itemize}
  \item Add your FPGA folder in {\tt ./hardware/fpga} using the other folders in there as examples
  \item In file {\tt ./system.mk}:
  \begin{enumerate}
    \item Define {\tt FPGA\_BOARD} with the name of your FPGA folder
    \item Define {\tt FPGA\_BOARD\_SERVER} with the URL or IP address of the computer connected to the board
    \item Define {\tt FPGA\_COMPILE\_SERVER} with the URL or IP address of the computer containing the FPGA compiler and tools
    \item Define {\tt FPGA\_BOARD\_ROOT\_DIR} and {\tt FPGA\_COMPILE\_ROOT\_DIR} with the name of the remote root directories for the repository files
  \end{enumerate}
  \item To compile and load the hardware design in the FPGA, type {\tt make fpga-load}
  \item To load and run your firmware in the FPGA, type {\tt make run-firmware}
\end{itemize}
\end{frame}

\begin{frame}{Implement in ASIC (WIP)}
\begin{itemize}
  \item Add your ASIC folder in {\tt ./hardware/asic} using the other folders in there as examples
  \item In the file {\tt ./system.mk}:
  \begin{enumerate}
     \item Define {\tt ASIC\_NODE} with the name of your ASIC folder
     \item Define {\tt ASIC\_COMPILE\_SERVER} with the URL or IP address of the computer containing the ASIC design tools
     \item Define {\tt ASIC\_COMPILER\_ROOT\_DIR} with the name of the remote root directory for the repository files
  \end{enumerate}
  \item To compile the ASIC, type {\tt make asic}
\end{itemize}
\end{frame}


\begin{frame}{Conclusions and future work}

\begin{itemize}
  \item Conclusions
    \begin{itemize}
    \item A tutorial on SoC creation using IOb-SoC is presented
    \item The addition of a peripheral IP core (timer) is illustrated
    \item A simple software driver for the IP core is exemplified
    \item How to compile and run the system is explained 
    \item Options for implementing the main memory are presented
    \item Implementation of FPGA and ASIC is explained
    \end{itemize}
  \item Future work
    \begin{itemize}
    \item Non-volatile (flash) external memory support
    \item Real Time Operating System (RTOS) 
    \end{itemize}
\end{itemize}

\end{frame}

% for including figures in slides:
%\begin{frame}{Introduction}
%\begin{center}
%  \begin{columns}[onlytextwidth]
%    \column{0.5\textwidth}
%    bla
%    \column{0.5\textwidth}
%    \begin{figure}
%      \centering
%       \includegraphics[width=0.9\textwidth]{turb.jpg}
%      \caption{1. Flow visualisation (source: www.bronkhorst.com).}
%      \label{fig:my_label}
%    \end{figure}
%  \end{columns}
%\end{center}
%\end{frame}

\end{document}
