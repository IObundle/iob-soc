\documentclass [xcolor=svgnames, t] {beamer} 
\usepackage[utf8]{inputenc}
\usepackage{booktabs, comment} 
\usepackage[absolute, overlay]{textpos} 
\useoutertheme{infolines} 
\setbeamercolor{title in head/foot}{bg=internationalorange}
\setbeamercolor{author in head/foot}{bg=dodgerblue}
\usepackage{csquotes}
\usepackage[style=verbose-ibid,backend=bibtex]{biblatex}
\bibliography{bibfile}
\usepackage{amsmath}
\usepackage[makeroom]{cancel}
\usepackage{textpos}
\usepackage{tikz}
\usepackage{listings}
\graphicspath{ {./figures/} }


\usetheme{Madrid}
\definecolor{myuniversity}{RGB}{0, 60, 113}
\definecolor{internationalorange}{RGB}{231, 93,  42}
 	\definecolor{dodgerblue}{RGB}{0, 119,202}
\usecolortheme[named=myuniversity]{structure}

\graphicspath{ {./figures/} }

\title[IOB-SoC Presentation]{IOB-SoC}
\subtitle{A RISC-V-based System on Chip}
\institute[IObundle Lda]{IObundle Lda.\\The Architecture for an Agile World}
\titlegraphic{\includegraphics[height=2.5cm]{Logo.png}}
\author[José T. de Sousa]{Jos\'e T. de Sousa}
\institute[IObundle Lda]{IObundle Lda}
\date{\today}


\addtobeamertemplate{navigation symbols}{}{%
    \usebeamerfont{footline}%
    \usebeamercolor[fg]{footline}%
    \hspace{1em}%
    \insertframenumber/\inserttotalframenumber
}

\begin{document}

\begin{frame}
 \titlepage   
\end{frame}

%%%%%%%%%%%%%%%%%%%%%%%%%%%%
\logo{\includegraphics[scale=0.2]{Logo.png}~%
}
%%%%%%%%%%%%%%%%%%%%%%%%%%

\begin{frame}{Outline}
\begin{center}
   \begin{itemize}
     \item Introduction
     \item Project setup
     \item Project editing guidelines
     \item Getting started with the timer IP hardware design
     \item Edit the hardware definitions file {\tt system.vh}
     \item Instantiate the timer IP in file {\tt rtl/src/system.v}
     \item Add the timer IP firmware
     \item Edit file {\tt firmware.c} to drive the timer IP
     \item Edit the Makefile to compile the timer IP firmware
     \item Setup RTL simulation using the Icarus Verilog simulator
     \item Simulate the system
     \item Conclusions and future work
 \end{itemize} 
\end{center}
\end{frame}


\begin{frame}{Introduction}
\begin{center}
    \begin{itemize}
      \item Building processor based systems from scratch is hard
      \item The IOB-SoC template eases this task
      \item Provides a base Verilog SoC equipped with
        \begin{itemize}
        \item a RISC-V CPU
        \item a memory system including a boot ROM and a RAM module
        \item a UART communication module
        \end{itemize}
      \item Users can add IP cores to build more complex SoCs
      \item Here, the addition of a timer IP is exemplified
    \end{itemize}
\end{center}
\end{frame}

\begin{frame}{Project setup}
\begin{center}
  \begin{itemize}
    \item Use a Linux machine virtual or not
    \item Install the latest stable version of the open source Icarus Verilog simulator
    \item Make sure you can access {\it github.com} using an ssh key
    \item At {\it github.com} create your SoC repository  called {\it my-soc} using {\it github.com/IObundle/iob-soc} as a template and cd into it
    \item Follow the instructions in the README file to clone the repository in your Linux machine
  \end{itemize}
\end{center}
\end{frame}


\begin{frame}{Create an IP core to instantiate in your SoC}
\begin{itemize}
\item Create a timer IP core or alternatively download the one from
  \\{\tt git clone git@bitbucket.org:jjts/iob-timer.git}
\item Edit the following lines of the system configuration file file {\tt ./system.mk} as in the next slide
\end{itemize}
\end{frame}

\begin{frame}[fragile]{Edit the system configuration file {\tt ./system.mk}}
\begin{tiny}
\begin{lstlisting}
#FIRMWARE
FIRM_ADDR_W:=13

#SRAM
SRAM_ADDR_W=13

#DDR
USE_DDR:=0
RUN_DDR:=0
DDR_ADDR_W:=30

#BOOT
USE_BOOT:=0
BOOTROM_ADDR_W:=12

#Number of slaves (peripherals)
N_SLAVES:=2

#Peripheral IDs (assign serially: 0, 1, 2, etc)
UART:=0
TIMER:=1
\end{lstlisting}
\end{tiny}
\end{frame}


\begin{frame}[fragile]{Instantiate the timer IP in file {\tt rtl/src/system.v}}
\begin{tiny}
\begin{lstlisting}
`timescale 1 ns / 1 ps
`include "system.vh"

module system (

   ...

   time_counter #(.COUNTER_WIDTH(32))
   timer (
          rst(reset_int),
          clk(clk),
          .addr(m_addr[2]),
          .data_in(m_wdata),
          .data_out(s_rdata[`TIMER_BASE]),
          .valid(s_valid[`TIMER_BASE]),
          .ready(s_ready[`TIMER_BASE])
    );

   ...
endmodule
\end{lstlisting}
\end{tiny}
\end{frame}

%\begin{frame}{Add the timer IP software}
%\end{frame}

\begin{frame}[fragile]{Edit file {\tt firmware.c} to drive the timer IP}

  Edit the {\tt ./software/firmware/firmware.c} file as in the next slide

\begin{tiny}
\begin{lstlisting}
#include "system.h"
#include "iob-uart.h"
#include "iob_timer.h"

#define UART (UART_BASE<<(DATA_W-N_SLAVES_W))
#define SOFT_RESET (SOFT_RESET_BASE<<(ADDR_W-N_SLAVES_W))
#define TIMER (TIMER_BASE<<(ADDR_W-N_SLAVES_W))

int main()
{ 

  //variable to read timer cycle count
  int cycles = timer_get_count(TIMER_BASE);
  
  uart_init(UART,UART_CLK_FREQ/UART_BAUD_RATE);   

  uart_printf("Hello world!\n");

  uart_txwait();
  
  //read current timer count and compute elapsed time in clock cycles  
  cycles =  timer_get_count(TIMER) - cycles;

  //print the elapsed time and clock frequency
  uart_printf("Execution time: %dus @%dMHz,115200BAUD\n",(time*1000000)/UART_CLK_FREQ);

  uart_txwait();
  return 0;
}
\end{lstlisting}
\end{tiny}
\end{frame}


\begin{frame}[fragile]{Edit the Makefile to compile the timer IP firmware}
\begin{tiny}
\begin{lstlisting}
TOOLCHAIN_PREFIX = riscv32-unknown-elf-
PYTHON_DIR := ../python/
SUBMODULES_DIR := ../../iob-soc/submodules
UART_DIR := $(SUBMODULES_DIR)/iob-uart/c-driver

#specify timer directory and addi it to include path 
TIMER_DIR := ../../iob-timer
INCLUDE = -I. -I$(UART_DIR) -I$(TIMER_DIR)
DEFINE = -DUART_BAUD_RATE=$(BAUD) -DUART_CLK_FREQ=$(FREQ)

#***NEW*** add timer.c to the source list
SRC =  firmware.S firmware.c $(UART_DIR)/iob-uart.c  $(TIMER_DIR)/iob_timer.c

all: firmware.lds $(SRC) system.h $(UART_DIR)/iob-uart.h
	$(TOOLCHAIN_PREFIX)gcc -Os -ffreestanding  -nostdlib -o firmware.elf $(DEFINE) $(INCLUDE) $(SRC) --std=gnu99 -Wl,-Bstatic,-T,firmware.lds,-Map,firmware.map,--strip-debug -lgcc -lc
	$(TOOLCHAIN_PREFIX)objcopy -O binary firmware.elf firmware.bin
	$(eval MEM_SIZE=`./get_firmsize.sh`)
	$(eval MEM_SIZE=`$(PYTHON_DIR)/get_memsize.py MAINRAM_ADDR_W`)
	$(PYTHON_DIR)/makehex.py firmware.bin $(MEM_SIZE) > firmware.hex

system.h: ../../rtl/include/system.vh
	sed s/\`/\#/g ../../rtl/include/system.vh > system.h

clean:
	@rm -rf firmware.bin firmware.elf firmware.map *.hex *.dat
	@rm -rf uart_loader system.h
	@rm -rf ../uart_loader

.PHONY: all clean
\end{lstlisting}
\end{tiny}
\end{frame}



\begin{frame}[fragile]{Setup RTL simulation using the Icarus Verilog simulator}

\begin{itemize}
\item Copy the simulation folder from the iob-soc repository\\ {\tt cp -r
  iob-soc/simulation .}
\item Edit file ``simulation/icarus/Makefile'' as below, to compile the system
  and simulate it
\end{itemize}

\begin{tiny}
\begin{lstlisting}
#simulation baud rate
BAUD := 1000000
FREQ := 100000000

#paths
ROOT_DIR := ../..
SUBMODULES_DIR := $(ROOT_DIR)/iob-soc/submodules
FIRM_DIR := $(ROOT_DIR)/software/firmware

(...)

#***NEW*** add timer directory to the hw include paths
TIMER_DIR := $(ROOT_DIR)/iob-timer
HW_INCLUDE := -I. -I$(RTL_DIR)/include -I$(UART_DIR)/rtl/include -I$(CACHE_DIR)/rtl/header -I$(TIMER_DIR)

#***NEW*** add the timer IP verilog source to the list of sources
VSRC = $(RTL_DIR)/testbench/system_tb.v  $(RTL_DIR)/src/*.v $(RTL_DIR)/src/memory/behav/*.v $(RISCV_DIR)/picorv32.v $(UART_DIR)/rtl/src/iob-uart.v $(FIFO_DIR)/afifo.v $(CACHE_DIR)/rtl/src/gen_mem_reg.v $(CACHE_DIR)/rtl/src/iob-cache.v $(AXI_RAM_DIR)/rtl/axi_ram.v $(TIMER_DIR)/iob_timer.v

(...)
\end{lstlisting}
\end{tiny}
\end{frame}

\begin{frame}[fragile]{Simulate the system}

\begin{itemize}
\item To simulate the system just type\\
  {\tt make} 
\item The firmware, bootloader and system verilog description are compiled as you can see from the printed messages
\item The last prints should look like the following
\end{itemize}

\begin{tiny}
\begin{lstlisting}
  ./a.out
  VCD info: dumpfile system.vcd opened for output.
  Hello world!
  Total execution time: 1262 us @100MHz
\end{lstlisting}
\end{tiny}

\begin{itemize}
\item Congratulations! You have created your first RISC-V system using IOB-SoC!!\\
\end{itemize}

\end{frame}



\begin{frame}{Main memory options}
%\begin{small}
\begin{itemize}
\item The main memory type options are enabled by defining or undefining the {\tt USE\_DDR} and {\tt USE\_BOOT} macros
\item Option 1: place the firmware image in FPGA memory during design compilation
  \begin{itemize}
  \item Do not define either macro: \underline{not reprogrammable}
  \item This option is only valid for FPGA: needs recompilation if firmware changes
  \end{itemize}
\item Option 2: place the firmware in internal RAM
  \begin{itemize}
  \item Define macro {\tt USE\_BOOT} only: \underline {reprogrammable}
  \item This option is valid for FPGA and ASIC
  \item Firmware can be (re)loaded via UART
  \end{itemize}
\item Option 3: place the firmware in external DDR memory
  \begin{itemize}
  \item Define both macros: \underline{reprogrammable}
  \item This option is valid for FPGA and ASIC
  \item Firmware can be (re)loaded via UART
  \item Third party DDR controller IP core is required
  \end{itemize}
\end{itemize}
\end{frame}
%\end{small}




\begin{frame}{Conclusions and future work}

\begin{itemize}
  \item Conclusions
    \begin{itemize}
    \item Presented project and editing guidelines
    \item Designed of a peripheral IP (timer)
    \item Instantiated peripheral in the system 
    \item Designed a simple software driver for the peripheral
    \item Compiled the software
    \item Simulated the system's RTL code running the software in the memories
    \item Presented options for the main memory
    \end{itemize}
  \item Future work
    \begin{itemize}
    \item Non volatile (flash) external memory support
    \item Real Time Operating System (RTOS) 
    \end{itemize}
\end{itemize}

\end{frame}




%\begin{frame}{Introduction}
%\begin{center}
%  \begin{columns}[onlytextwidth]
%    \column{0.5\textwidth}
%    bla
%    \column{0.5\textwidth}
%    \begin{figure}
%      \centering
%       \includegraphics[width=0.9\textwidth]{turb.jpg}
%      \caption{1. Flow visualisation (source: www.bronkhorst.com).}
%      \label{fig:my_label}
%    \end{figure}
%  \end{columns}
%\end{center}
%\end{frame}

\end{document}
