\documentclass [xcolor=svgnames, t] {beamer} 
\usepackage[utf8]{inputenc}
\usepackage{booktabs, comment} 
\usepackage[absolute, overlay]{textpos} 
\useoutertheme{infolines} 
\setbeamercolor{title in head/foot}{bg=internationalorange}
\setbeamercolor{author in head/foot}{bg=dodgerblue}
\usepackage{csquotes}
\usepackage[style=verbose-ibid,backend=bibtex]{biblatex}
\bibliography{bibfile}
\usepackage{amsmath}
\usepackage[makeroom]{cancel}
\usepackage{textpos}
\usepackage{tikz}
\usepackage{listings}
\graphicspath{ {./figures/} }
\usepackage{hyperref}
\hypersetup{
    colorlinks=true,
    linkcolor=blue,
    filecolor=magenta,      
    urlcolor=cyan,
}

\usetheme{Madrid}
\definecolor{myuniversity}{RGB}{0, 60, 113}
\definecolor{internationalorange}{RGB}{231, 93,  42}
 	\definecolor{dodgerblue}{RGB}{0, 119,202}
\usecolortheme[named=myuniversity]{structure}

\graphicspath{ {./figures/} }

\title[IOB-SoC Presentation]{IOB-SoC}
\subtitle{A RISC-V-based System on Chip}
\institute[IObundle Lda]{IObundle Lda.\\The Architecture for an Agile World}
\titlegraphic{\includegraphics[height=2.5cm]{Logo.png}}
\author[José T. de Sousa]{Jos\'e T. de Sousa}
\institute[IObundle Lda]{IObundle Lda}
\date{\today}


\addtobeamertemplate{navigation symbols}{}{%
    \usebeamerfont{footline}%
    \usebeamercolor[fg]{footline}%
    \hspace{1em}%
    \insertframenumber/\inserttotalframenumber
}

\begin{document}

\begin{frame}
 \titlepage   
\end{frame}

%%%%%%%%%%%%%%%%%%%%%%%%%%%%
\logo{\includegraphics[scale=0.2]{Logo.png}~%
}
%%%%%%%%%%%%%%%%%%%%%%%%%%

\begin{frame}{Outline}
\begin{center}
   \begin{itemize}
     \item Introduction
     \item Project setup
     \item Create an IP core to instantiate in your SoC
     \item Edit the ./system.mk configuration file to declare a new peripheral
     \item Instantiate the timer IP in file {\tt rtl/src/system.v}
     \item Edit file {\tt firmware.c} to drive the new peripheral
     \item Run the firmware in internal SRAM
     \item Run the firmware in external DDR
     \item Simulate and implement the system
     \item Conclusions and future work
 \end{itemize} 
\end{center}
\end{frame}


\begin{frame}{Introduction}
\begin{center}
    \begin{itemize}
      \item Building processor-based systems from scratch is challenging
      \item The IOB-SoC template eases this task
      \item Provides a base Verilog SoC equipped with
        \begin{itemize}
        \item a RISC-V CPU
        \item a memory system including boot ROM, RAM and AXI4 interface to DDR
        \item a UART communications module
        \end{itemize}
      \item Users can add IP cores and software to build more complex SoCs
      \item Here, the addition of a timer IP and its software driver is exemplified
    \end{itemize}
\end{center}
\end{frame}

\begin{frame}{Project setup}
\begin{center}
  \begin{itemize}
    \item Use a Linux machine or VM
    \item Install the latest stable version of the open source Icarus Verilog simulator (\url{iverilog.icarus.com})
    \item Make sure you can access \url{github.com} using an ssh key
    \item At \url{github.com} create your SoC repository using \url{github.com/IObundle/iob-soc} as a template
    \item Follow the instructions in the README file to clone the repository in your Linux machine
  \end{itemize}
\end{center}
\end{frame}


\begin{frame}{Create an IP core to instantiate in your SoC}
  \begin{itemize}
  \item Create a timer IP core repository or, alternatively, use the one at \url{github.com:IObundle/iob-timer.git}
  \item An IP core can be integrated in an IOb-SoC if it provides the following 2 files: 
    \begin{enumerate}
    \item hardware/hardware.mk
    \item software/embedded/embedded.mk
    \end{enumerate}
  \item Add the IP core repository as a git submodule of your IOb-SoC repository:\\
    {\tt git submodule add git@github.com/IObundle/iob-timer.git submodules/TIMER}
  \item To configure the system to host the IP core, edit the {\tt ./system.mk} file as in the next slide
  \end{itemize}
\end{frame}

\begin{frame}[fragile]{Edit the {\tt ./system.mk} configuration file to declare a new peripheral}
\begin{tiny}
\begin{lstlisting}
#FIRMWARE
FIRM_ADDR_W:=13

#SRAM
SRAM_ADDR_W=13

#DDR
USE_DDR:=0
RUN_DDR:=0
DDR_ADDR_W:=30

#BOOT
USE_BOOT:=0
BOOTROM_ADDR_W:=12

#Peripheral list (must match respective submodule name)
PERIPHERALS:=UART TIMER
\end{lstlisting}
\end{tiny}
\end{frame}


\begin{frame}[fragile]{Instantiate the timer IP in file {\tt hardware/src/system.v}}
\begin{tiny}
\begin{lstlisting}
`timescale 1ns/1ps
`include "system.vh"

module system (

   ...

   time_counter #(.COUNTER_WIDTH(32))
   timer (
          rst(reset_int),
          clk(clk),
          .addr(m_addr[2]),
          .data_in(m_wdata),
          .data_out(s_rdata[`TIMER_BASE]),
          .valid(s_valid[`TIMER_BASE]),
          .ready(s_ready[`TIMER_BASE])
    );

   ...
endmodule
\end{lstlisting}
\end{tiny}
\end{frame}

%\begin{frame}{Add the timer IP software}
%\end{frame}

\begin{frame}[fragile]{Edit the {\tt firmware.c} file to drive the new peripheral}

  {\tt ./software/firmware/firmware.c}
  \begin{tiny}
    \begin{lstlisting}
#include "system.h"
#include "iob-uart.h"
#include "iob_timer.h"

int main()
{ 

  //read timer cycle count
  int cycles = timer_get_count(TIMER_BASE);
  
  uart_init(UART,UART_CLK_FREQ/UART_BAUD_RATE);   

  uart_printf("Hello world!\n");

  uart_txwait();
  
  //read current timer count and compute elapsed time in clock cycles  
  cycles =  timer_get_count(TIMER) - cycles;

  //print the elapsed time and clock frequency
  uart_printf("Execution time: %dus @%dMHz,115200BAUD\n",(time*1000000)/UART_CLK_FREQ);

  uart_txwait();
  return 0;
}
\end{lstlisting}
\end{tiny}
\end{frame}


\begin{frame}{Run the firmware in internal SRAM}
\begin{enumerate}
\item Run the firmware in internal RAM and disable (re)programming
  \begin{itemize}
  \item Assign {\tt USE\_DDR=0} and {\tt USE\_BOOT=0}
  \item Loading programs after the FPGA is programmed is disabled: if the firmware is modified the FPGA must be recompiled
  \item This option is only valid for FPGA which permits memory initialisation
  \end{itemize}
\item Run the firmware in internal RAM and enable (re)programming
  \begin{itemize}
  \item Assign {\tt USE\_DDR=0} {\tt USE\_BOOT=1}
  \item Loading programs after the FPGA is programmed is enabled
  \item This option is valid for FPGA and ASIC
  \item Firmware is (re)loaded via UART
  \end{itemize}
\end{enumerate}
\end{frame}

\begin{frame}{Run the firmware in external DDR}
\begin{enumerate}
\item Run the firmware in external DDR and disable (re)programming
  \begin{itemize}
  \item Assign {\tt USE\_DDR=1} and {\tt USE\_BOOT=0}
  \item This option is only allowed in simulation which permits memory initialisation
  \item An FPGA or ASIC implementation will not work
  \end{itemize}
\item Run the firmware in external DDR memory and enable (re)programming
  \begin{itemize}
  \item Define {\tt USE\_DDR=1} {\tt USE\_BOOT=1}
  \item This option is valid for FPGA and ASIC
  \item Firmware is (re)loaded via UART
  \item Third party DDR controller IP core is required
  \end{itemize}
\end{enumerate}
\end{frame}

\begin{frame}[fragile]{Simulate and implement the system}

\begin{itemize}
\item To simulate the system just type {\tt make} 
\item The firmware, bootloader and system verilog description are compiled as you can see from the printed messages
\item The last prints should look like the following
\end{itemize}

\begin{tiny}
\begin{lstlisting}

IOb-SoC Bootloader:

Reboot CPU and run program...

Hello world!
Total execution time: 1262 us @100MHz
\end{lstlisting}
\end{tiny}

\begin{itemize}
\item To implement in your chosen FPGA just type {\tt make fpga}
\item To implement in your chosen ASIC just type {\tt make asic}
\item To load the firmware in the hardware just type {\tt make load-firmware}
\end{itemize}

\end{frame}

\begin{frame}{Conclusions and future work}

\begin{itemize}
  \item Conclusions
    \begin{itemize}
    \item A tutorial on SoC creation using IOb-SoC is presented
    \item The addition of a peripheral IP core (timer) is illustrated
    \item A simple software driver for the IP core is exemplified
    \item How to compile and run the system is explained 
    \item Options for implementing the main memory are presented
    \end{itemize}
  \item Future work
    \begin{itemize}
    \item Non-volatile (flash) external memory support
    \item Real Time Operating System (RTOS) 
    \end{itemize}
\end{itemize}

\end{frame}

% for including figures in slides:
%\begin{frame}{Introduction}
%\begin{center}
%  \begin{columns}[onlytextwidth]
%    \column{0.5\textwidth}
%    bla
%    \column{0.5\textwidth}
%    \begin{figure}
%      \centering
%       \includegraphics[width=0.9\textwidth]{turb.jpg}
%      \caption{1. Flow visualisation (source: www.bronkhorst.com).}
%      \label{fig:my_label}
%    \end{figure}
%  \end{columns}
%\end{center}
%\end{frame}

\end{document}
