\documentclass{ug}
\usepackage{xltabular}

%replace ipcore-name by the name of your ip core (e.g. IOb-Cache) and description by a brief description (e.g. a Configurable Cache)

\title{%
\Huge ipcore-name \\
 \vspace*{3cm}
\Large description
}

\header{ipcore-name, description}

\date{\today}
\category{User Guide, \input{\NAME_version.tex}, Build \input{shortHash.tex}}

\definecolor{iob-green}{rgb}{0.0,1.0,0.80}
\definecolor{iob-blue}{rgb}{0.90196,0.94902,1}
\definecolor{dark-blue}{RGB}{155,194,230}
\definecolor{mid-blue}{RGB}{189,215,238}
\definecolor{light-blue}{RGB}{221,235,247}


\begin{document}

\maketitle
\pagenumbering{gobble}

\vspace*{\fill}
User Guide, \input{\NAME_version.tex}, Build \input{shortHash.tex}
\hspace*{\fill} \includegraphics[keepaspectratio,scale=.7]{Logo.png}

\cleardoublepage
\pagenumbering{roman}
\setcounter{page}{1}
\today & Initial document version \input{version}. \\ \hline

\cleardoublepage
\tableofcontents
\clearpage
\listoftables
\clearpage
\listoffigures
\cleardoublepage
\pagenumbering{arabic}
\setcounter{page}{1}
\section{Introduction}
\label{sec:intro}
IOb-SoC is a RISC-V-based System-on-Chip Platform written in Verilog, which
users can download for free, modify, simulate and implement in FPGA or ASIC. It
supports stand-alone and boot loading modes, and can use an internal RAM or an
external DDR controller via an L1/L2 cache system. The IP is currently supported
in ASICs and FPGAs. Licensable commercial versions are available.

\begin{figure}[!htbp]
  \centerline{\includegraphics[width=14cm]{symb.pdf}}
  \vspace{0cm}
  \caption{IP Core Symbol}
  \label{fig:symbol}
\end{figure}


\subsection{Features}
\label{sec:feat}
% SPDX-FileCopyrightText: 2024 IObundle
%
% SPDX-License-Identifier: MIT

\begin{itemize}
\item 32-bit RISC-V control CPU
\item Support for Integer (I), atomic (A), and multiply/divide extensions (M)
\item Instruction and data caches
\item RS232 interfaces for viewing runtime messages
\item Optional timer peripheral
\item Optional Ethernet peripheral
\item Frequency of operation at 167MHz on FPGA
\item Needs external DDR4 memory controller IP
\end{itemize}


\subsection{Benefits}
\label{sec:benef}
% SPDX-FileCopyrightText: 2024 IObundle
%
% SPDX-License-Identifier: MIT

\begin{itemize}
  \itemsep-0.5em
\item Compact and easy to integrate hardware and software implementation
\item Can fit many instances in low-cost FPGAs
\item Low power consumption
\end{itemize}


\subsection{Deliverables}
\label{sec:deliv}
\begin{itemize}
  \itemsep-0.5em
\item FPGA synthesized netlist or Verilog source code, and respective
  synthesis and implementation scripts
\item FPGA verification environment by simulation and emulation
\item Bare-metal software driver and example user software
\item User documentation for easy system integration
\end{itemize}


\section{Description}

This section gives a detailed description of the IP core. The high-level block
diagram is presented, along with a description of its blocks. The parameters and
macros that define the core configuration are listed and explained. The
interface signals are enumerated and described; if timing diagrams are needed,
they are shown after the interface signals. Finally the Control and Status
Registers (CSR) are outlined and explained.

\subsection{Block Diagram}
\label{sec:bdd}
Figure~\ref{fig:bd} presents a high-level block diagram of the core, followed by
a brief description of each block.

\begin{figure}[H]
  \centering {\includegraphics[width=\columnwidth]{bd.pdf}}
  \vspace{-0.7cm}
  \caption{High-Level Block Diagram}
  \label{fig:bd}
\end{figure}


\setlist[description]{font=\bfseries\MakeUppercase, labelindent=\parindent, leftmargin=*}

\input{blocks}


\ifdefined\CONFS
\subsection{Configuration}
\label{sec:ipconfig}
This section describes how the IP core can be configured by means of
Table~\ref{tab:confs}. The core may be configured using parameters or
macros. The parameters are passed to each instance of the core and only affect
that instance. The macros apply to all instances of the core. The macros and
parameters have the following types:

\begin{description}
\item \textbf{'M'} Valued Macro: the value of the macro dictates the way the core is built.
\item \textbf{'P'} True Parameter: the value of the parameter influences how the core instance that receives is built.
\item \textbf{'F'} False Parameter: the parameter needs to be on the parameter list, but cannot be changed by the user.
\end{description}

\begin{xltabular}{\textwidth}{|l|c|c|c|c|X|} \hline
    \rowcolor{iob-green}
    {\bf Macro} & {\bf Type} & {\bf Min} & {\bf Typical} & {\bf Max} & {\bf Description}
    \\ \hline \hline
    \input confs_tab
    \caption{Configuration Macros.}\label{tab:confs}
\end{xltabular}

Note: The 16-bit auto-created Product Version macro uses nibbles to represent
decimal numbers using their binary values.  The two most significant nibbles
represent the integral part of the version, and the two least significant
nibbles represent the decimal part.  For example V12.34 is represented by
0x1234.

\fi

\subsection{Interface Signals}
\label{sec:ifsig}
\begin{table}[H]
  \centering
  \begin{tabular}{|l|l|r|p{9.5cm}|}
    
    \hline
    \rowcolor{iob-green}
    {\bf Name} & {\bf Direction} & {\bf Width} & {\bf Description}  \\ \hline \hline

    \input{rs232_if_tab}
 
  \end{tabular}
  \caption{RS232 Interface Signals}
  \label{rs232_is_tab:is}
\end{table}

\input{\TEX/ug/iob_s_if}

%TODO
%\input{\TEX/ug/axil_s_if}


%timing diagrams
\ifdefined\TD
\subsection{Timing Diagrams}
\label{sec:td}
\input{td}
\fi

%software components
\ifdefined\SWREG
\subsection{Control and Status Registers}
\label{sec:swreg}
\begin{table}[H]
  \centering
  \begin{tabular}{|l|c|c|r|P{0.8cm}|p{5.5cm}|}
    \hline
    \rowcolor{iob-green}
    
    {\bf Name} & {\bf R/W} & {\bf Addr} & {\bf Bits} & {\bf Initial Value} & {\bf Description} \\ \hline

    \input{sw_uartreg_tab}
    
  \end{tabular}
  \caption{UART software accessible registers.}
  \label{tab:sw_tx}
\end{table}


\fi

\section{Usage}

\subsection{Instantiation}
\label{sec:inst}
Figure~\ref{fig:inst} illustrates how to instantiate the IP core and, if applicable, the
required external blocks.

\begin{figure}[!htbp]
    \centerline{\includegraphics[width=14cm]{inst.pdf}}
    \vspace{0cm}\caption{Core Instance and Required Surrounding Blocks}
    \label{fig:inst}
\end{figure}

Add here any specific ip core description.




\subsection{Simulation}
\label{sec:tbbd}
% SPDX-FileCopyrightText: 2024 IObundle
%
% SPDX-License-Identifier: MIT

The provided testbench implements a self-loop, where the IOb-UART handshakes
({\tt cts} to {\tt rts}, and sends data to itself ({\tt txd} to {\tt rxd}). The
testbench drives the clock and reset signals, and emulates the CPU actions with
a simple control block.


\ifdefined\ASICSYNTH
\subsection{ASIC Synthesis}
\label{sec:synth}
% SPDX-FileCopyrightText: 2024 IObundle
%
% SPDX-License-Identifier: MIT

This IP core synthesizes for FPGA and ASIC with very low logic resources
consumption. The implementation does not require memory or arithmetic resources.

\fi

\ifdefined\FPGACOMP
\subsection{FPGA Compilation}
\label{sec:fpga}
The Quartus and Vivado FPGA compilation toolchains are supported via {\tt .tcl}
scripts invoked by a Makefile. The script compiles and elaborates the design for
a given set of target FPGA boards. Additional boards can be supported by
following the flow provided.

The core is instantiated within a Verilog wrapper, which depends on the
specifics of the FPGA device and the FPGA board. Timing constraints file(s) are
provided.

After compilation, reports on FPGA resource usage, power consumption, and timing
closure are generated. A post-synthesis Verilog file is created, which can be
used in post-synthesis simulation.

\input{fpga_desc}

\subsubsection*{System-level FPGA Run}

Upon request, files to run the core on the supported FPGA boards can be
provided. The core is embedded in a RISC-V system and exercised in various
modes, using a bare-metal software program written in the C programming
language.

\fi

\ifdefined\RESULTS
\section{Implementation Results}
\label{sec:results}
\ifdefined\FPGA
\subsection{FPGA}
\input{fpga_results}
\fi


\ifdefined\ASIC
\subsection{ASIC}
\ifdefined\ASIC
\begin{table}[H]
\centering
\begin{tabular}{|c|c|c|c|}
\hline
\rowcolor{iob-green}
\textbf{Area} ($\SI{}{\micro\meter\squared}$)  & \textbf{Gates}  & \textbf{FFs}  & \textbf{Frequency} ($\SI{}{\mega\hertz}$)\\
\hline
\hline
\input asic.tex
\end{tabular}
\caption{ASIC results for node UMC 130nm.}
\label{tab:asic_results}
\end{table}
\fi

\fi

\fi

\ifdefined\CUSTOM
\input{custom}
\fi

\end{document}
